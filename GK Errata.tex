\documentclass[]{article}
\usepackage{color}
\usepackage[normalem]{ulem}
\usepackage{array}
\usepackage{booktabs}
\usepackage{amsmath}
\usepackage{amsfonts}
\usepackage{amssymb}


%opening
\title{Clarification and Errata to \\\textit{Catastrophe Modeling: \\A New Approach to Managing Risk} \\(Grossi, P. and Kunreuther, H., Editors)}
\author{Casualty Actuarial Society Syllabus Committee \thanks{Prepared by Rajesh Sahasrabuddhe, FCAS, MAAA, CAS Syllabus Committee Chairperson} } %\\with assistance from Dave Clark, FCAS, MAAA}


\begin{document}

\maketitle

\begin{abstract}

\end{abstract}
This notes presents an errata and clarifying remarks to Section 2.4 Derivation and Use of an Exceedance Probability Curve of \textit{Catastrophe Modeling: A New Approach to Managing Risk}.
\section{Clarification}
The use of the phrase ``exceedance probability" in Section 2.4 is ambiguous.  Specifically, “exceedance probability” can be used in one of three ways:
\begin{description}
\item[Occurrence Exceedance Probability (OEP)] The OEP is the probability that at least one loss exceeds the specified loss amount.
\item[Aggregate Exceedance Probability (AEP)] The AEP is the probability that the sum of all losses during a given period exceeds some amount.
\item[Conditional Exceedance Probability (CEP)] The CEP is the probability that the amount on a single event exceeds a specified loss amount; this is equal to 1-CDF of the severity curve as used by actuaries in other contexts.
\end{description}

For actuaries who have not worked with catastrophe models, the OEP may be a new concept.  Actuaries usually think of severity distributions, which correspond to the CEP - not the OEP.  In Section 2.4, the term ``exceedance probability" refers to the \textbf{Occurrence Exceedance Probability (OEP)}. The $OEP$ is the distribution of the largest loss in the period and is based on the theory of order statistics.

\section{Errata}
\begin{itemize}
\item The first complete paragraph on page 30 is corrected as follows:\\
\\
The events listed in Table 2.1 are assumed to be independent Bernoulli random variables \color{red} .\sout{, each with a}.  It is assumed that each event only occurs once with \color{red} the \color{black} probability mass function defined as: \ldots   

\item The second complete paragraph on page 30 is corrected as follows:\\
\\
If an event $E_i$ does not occur, the loss \color{red} for that event \color{black} is 0. \ldots 

\item The third complete paragraph on page 30 is corrected as follows:\\
\\
The \color{red} \sout{overall expected loss} average size of an individual loss \sout{for the entire series of events}\color{black}, denoted as the \color{red} \sout{average annual loss (AAL)}Average Loss Event ($ALE$) \color{black} in Table 2.1, is the sum of expected losses \color{red} \sout{of} for \color{black} each of the individual events for a given year and is given by:\\
\begin{equation*}
 \color{red}ALE \color{black} = \sum_i p_i \times L_i  \\
\end{equation*}
\color{red} Note: Candidates should review the definition of average annual loss (AAL) in the Glossary as this term is used in other sections of this material.
\color{black}

\item The fourth complete paragraph on page 30 is corrected as follows:\\
\\
Assuming that during a given year, only one \color{red} of each \color{black} disaster occurs, the \color{red} $OEP$  \sout{exceedance probability} \color{black} for a given level of loss, $\color{red}O \color{black}EP(L_i)$, can be determined by calculating: \ldots
\item The first sentence of the fifth complete paragraph on page 30 is corrected as follows:\\
\\
The resulting \color{red} $OEP$ is the probability that at least one loss exceeds a given value  \sout{exceedance probability is the annual probability that the loss exceeds a given value}\color{black}.
\newpage
\item Table 2.1 is replaced with the following:
% Table generated by Excel2LaTeX from sheet 'Sheet1'
\numberwithin{table}{section}
\begin{table}[h]
  \centering
  \caption{Events, Losses and Probabilities}
    \begin{tabular}{crrrr}
    \toprule
%Line 1
    
 & \multicolumn{1}{c}{Annual} & & \multicolumn{1}{c}{Occurrence}
 &   \\

%Line 2
 & \multicolumn{1}{c}{Probability of } & 
 & \multicolumn{1}{c}{Exceedance} &  \\

%Line 3
Event & \multicolumn{1}{c}{Occurrence} & \multicolumn{1}{c}{Loss}
 & \multicolumn{1}{c}{Probability} & \multicolumn{1}{c}{$E[L] =$ } \\

%Line 4
\multicolumn{1}{c}{$(E_i)$} & \multicolumn{1}{c}{$(p_i)$} & \multicolumn{1}{c}{$(L_i)$}
 & \multicolumn{1}{c}{$[OEP(L_i)]$} & \multicolumn{1}{c}{$p_i \times L_i$ } \\


    \midrule
    1     & 0.002 &         \$25,000,000  & 0.0000 &            \$50,000  \\
    2     & 0.005 &          15,000,000  & 0.0020 &            75,000  \\
    3     & 0.010 &          10,000,000  & 0.0070 &          100,000  \\
    4     & 0.020 &            5,000,000  & 0.0169 &          100,000  \\
    5     & 0.030 &            3,000,000  & 0.0366 &            90,000  \\
    6     & 0.040 &            2,000,000  & 0.0655 &            80,000  \\
    7     & 0.050 &            1,000,000  & 0.1029 &            50,000  \\
    8     & 0.050 &                800,000  & 0.1477 &            40,000  \\
    9     & 0.050 &                700,000  & 0.1903 &            35,000  \\
    10    & 0.070 &                500,000  & 0.2308 &            35,000  \\
    11    & 0.090 &                500,000  & 0.2847 &            45,000  \\
    12    & 0.100 &                300,000  & 0.3490 &            30,000  \\
    13    & 0.100 &                200,000  & 0.4141 &            20,000  \\
    14    & 0.100 &                100,000  & 0.4727 &            10,000  \\
    15    & 0.283 &                            0    & 0.5255 &                     0    \\
     \midrule
    Total & 1.000 & \multicolumn{2}{r}{Average Loss Event} &          760,000  \\
    \bottomrule
    \end{tabular}%
  \label{tab:addlabel}%
\end{table}%


\end{itemize}



\end{document}
